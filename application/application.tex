\documentclass{article}
\usepackage[utf8]{inputenc}

\begin{document}
\section*{Developing an FPGA Based Robot for Swarm Research}
Research in swarming behaviour is often done exclusively in simulation. 
This is a problem because it ignores the embodied properties of the animals we are trying to replicate. 
In this project a small robot will be developed. 
A former project listed the requirements for building a swarm of ten robots. 
It was found that communication using sound was the most suitable solution and that implementing the signal processing at sufficient speeds would be challenging.\\

In this project the existing robot platform will be converted to run on a Xilinx Zynq processor. 
The advantage of this is that sound processing can be parallelised to a large extend, but at the cost of added complexity. 
The goal is to encapsulate the complex programmable logic in configurable IP cores and write the necessary linux drivers to allow the robot to be operated as a linux device.\\

The output of the project is a robot and a written report in English.

\subsection*{Learning goals}  

\paragraph*{Knowledge:}~\\
\begin{itemize}
	\item Extensive understanding of the collaboration between PL and PS in the Zynq chip.
	\item Extensive knowledge about how to structure and write linux drivers for embedded electronics.
\end{itemize}
  
\paragraph*{Skills:}~\\
\begin{itemize}
	\item Able to use the Vivado suite and design flow to create advanced solutions in VHDL and C.
	\item Able to develop and test software integrated in linux based systems.
\end{itemize}
 
\paragraph*{Competencies:}~\\
\begin{itemize}
	\item Analysing constraints and requirements for complex embedded systems.
	\item Disseminating desicions and findings in a interdisciplinary research project.
\end{itemize}

\end{document}