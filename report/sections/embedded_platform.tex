%!TEX root = ../main.tex
\subsection{Embedded Platform} % (fold)
\label{sub:embedded_platform}
Previously a Raspberry Pi was used to power the platform.
As mentioned, it has been requested that the system is ported to a Zynq platform.
Currently available to the authors are the Zybo and the Zynqberry platforms.
Both are based around the Zynq-7010 chip.
A number of GPIO ports are required; four pins for PWM for the motors, four pins for encoders, four pins for microphones, one for a piezo transducer.
This amounts to a minimum of 13 pins required.
Potentially, an additional number of pins may be used to niterface to debugging LED's.
Below is an overview of the pros and cons of the two platforms.
\begin{itemize}
	\item \textbf{Zynqberry:} As the name implies, this platform is made to conform with the physical layout of the Raspberry Pi.
	Using this platform would allow the reuse of the mounting solution developed for the Raspberry Pi.
	At 26 total GPIO pins, the Zynqberry can supply the required GPIO\@.
	\item \textbf{Zybo:} This platform is significantly larger than the Zynqberry.
	The different form factor requires that a new mounting solution is devised.
	Additionally, the increased size means that the board will have to be mounted above the tracks to avoid interference.
	While this is an inconvenience, it should not pose an issue in collisions as the edges of the board are still within the bounds of the robot.
	At 48 total GPIO pins, the Zybo can supply the required GPIO\@.
	\item \textbf{MicroZed:}
	This platform is within the confines of the robot platform, but has a different formfactor than the Raspberry Pi and therefore the mounting solution would need a re-design.
	It can be used as a stand-alone evaluation board, but needs to be combined with a carrier card if the PLs I/O pins are to be used.
	This is because the I/O banks on the Zynq chip needs external powering, that the carrier card should supply.
	The I/O pins on the MicroZed are accessed through two 100-pin micro headers.
	The company behind the MicroZed also ships different carrier boards that can be used.
	If a custom carrier board is needed circuitry for control signals and power needs to be developed.

\end{itemize}
Both platforms are based around the same chip and should therefore be similar in functionality.
The authors, however, do have prior knowledge with the Zybo platform.
An up-to-date Linux system has been made to function with the Zybo.

\subsubsection*{MicroZed Carrier Card}
This section will define the requirements for a MicroZed carrier card.
MicroZed provides a carrier card design guide~\cite{design_carrier} that describes the requirements.
It describes how to utilize the Zynq chip on the MicroZed board.
In this project only a number of general I/Os and the two ADCs are needed.

\begin{itemize}
	\item Anti-aliasing filters for ADC inputs
	\item Power for MicroZed board and Zynq I/O banks 34 and 35.
	\item Correct power sequencing on boot
	\item Passthrough of Zynq bank I/Os
	\item Microheader connection to MicroZed
\end{itemize}

% subsection embedded_platform (end)
