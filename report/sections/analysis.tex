%!TEX root = ../main.tex

\section{Introduction}
A platform for research of flocking behaviour in swarms of robots was developed throughout the bachelor's thesis, ``Investigating Bio-inspired Object Avoidance in a Swarm of Mobile Robots''.
\thomas{insert names of students?}
Throughout that project an analysis was carried out in order to determine the requirements of the system.
This project aims to replace the Raspberry Pi with an FPGA platform, namely a Zynq platform.
Using an FPGA/ARM combination is believed to better enable the use of swarm algorithms.
Additionally, since the completion of the bachelor's thesis, a new type of microphone has been procured.
This type is digital, as opposed to the previous analogue microphones.
Much of the electronics developed for that project is developed so as to work around the shortcomings of the Raspberry Pi, as well as the amplifier circuits required for the microphones.
This project will redesign the electronics where necessary in order to accommodate the changes on the platform.

\section{Analysis} % (fold)
\label{sec:analysis}
This analysis will seek to expand on the analysis carried out in the aforementioned thesis.
Some of the conclusions reached are no longer valid due to the new platform.
It is necessary to determine which parts will need redesign and possibly, what new features will need to be added altogether.

\subsection{Mechanical Platform} % (fold)
\label{sub:mechanical_platform}
The chassis, battery and motors, including their encoders remain unchanged and as such will not be discussed further in this context.
% subsection mechanical_platform (end)

\subsection{Microphones} % (fold)
\label{sub:microphones}
As mentioned, a new set of digital microphones has been procured for use with this robot.
The previous electronic circuits developed include an amplifier section for the analogue microphones.
This is no longer necessary.
The analysis did find, however, that the multilateration algorithm is more robust when the microphones are further apart, therefore the new microphones should be placed towards the corners, as was done with the previous microphones.
% subsection microphones (end)

\subsection{Click Generator} % (fold)
\label{sub:click_generator}
Previously, it was attempted to generate a click using a piezo transducer.
The attempt did not manage to produce a sufficiently loud click.
A piezo transducer deforms when a voltage is applied to it.
A higher voltage increases the deformation.
By repeatedly pulsing the transducer with a sufficiently high voltage, it should be possible to generate a sound.
Some type of circuit will have to be used to generate the necessary voltage spike.

The sound of a Piezo transducer is amplified when placed in a plastic housing with a small hole to let the soundwaves out.
This housing and the size of the hole should be designed to amplify the specific piezo transducer's resonant freqency.
Designing such a housing is infeasable, considering how cheaply a finished product can be procured.
A piezo element is generally designed to generate sound with a specific frequency.
This can be seen by inspection of the frequency response curve of a piezo transducer.
A depiction of a typical response curve can be seen in figure \ref{fig:piezo_response}

\begin{figure}[H]
	\centering
	\includegraphics[width=.5\linewidth]{graphics/piezo_response.png}
	\caption{Typical frequency response of a piezo electric transducer. This particular model is most effective at approximately 5kHz and 18kHz.}
	\label{fig:piezo_response}
\end{figure}

Generally there are a range of frequencies around the resonant frequency that are at a much higher sound level than the rest of the frequencies.
Frequencies not in close range of the resonant frequencies are often attenuated by decades of dB.
This makes a normal piezo unfit for generating a click sound as this is approximated by a short frequency sweep.
Another option is the piezo speaker that is designed to have the same flat frequency response curve as an ordinary speaker.
It is therefore more fit to produce a clicking sound by making a frequency sweep.
An ordinary speaker can also be used, but generally they consume much more current to produce the same sound.
It was choosen to order three different kinds of sound generators to experiment with sound loudness, clicking sounds and power consumption.
\begin{itemize}
  \item Piezo element with plastic housing and internal drive circuit:\\
        \url{http://dk.rs-online.com/web/p/magnetisk-brummer-komponenter/5117664/}
  \item Piezo speaker in plastic housing without drive circuit: \\
        \url{http://dk.rs-online.com/web/p/piezoelektriske-miniaturehojttalere/2945630/}
  \item Miniature speaker:\\
          \url{http://dk.rs-online.com/web/p/miniature-hojttalere/7564595/}
\end{itemize}
% subsection click_generator (end)
\mikkel{Links directly in text?}
\subsubsection*{Piezo Element with Internal Drive Circuit}
This element produces a sound with a constant frequency just by connecting it to a DC power supply. 
This is very simply and allows for a minimal interface, but the internal drive circuit makes it impossible to produce a frequency sweep. 

\subsubsection*{Piezo Speaker without Internal Drive Circuit}
\thomas{Clearly, we developed a circuit for this, does that mean we chose this? needs some more explanation}
This Piezo speaker is designed to have a more flat frequency response and is therefore suitable for producing a frequency sweep. 
An external drive circuit needs to be developed as a drive circuit is not included with this product.
The developed circuit can be seen in figure \ref{fig:buzzer_circuit}.

\begin{figure}[h]
	\centering
	\includegraphics[width=.6\linewidth]{graphics/buzzer_circuit.pdf}
	\caption{Schematic of drive circuit for piezo speaker.}
	\label{fig:buzzer_circuit}
\end{figure}

A mosfet controlled by a \texttt{PWM} pin is toggling current from \texttt{VBATT} to \texttt{GND} and thereby controlling the sound generated by the piezo.
By changing the dutycycle of the \texttt{PWM} signal the volume produced can be controlled. 
A 50\% dutycycle yields the highest volume. 
A resistor is placed in parallel with the piezo speaker as it is capacitive and needs to discharge when the mosfet goes off.
The resistor in front of the mosfet is needed to limit the current from the \texttt{PWM} port. 

In experiments with the circuit and the piezo speaker, the \texttt{PWM} port was connected to a frequency generator set up to do frequency sweeps.
This produced loud and clear clicks.

\subsubsection*{Miniature speaker}
The miniature speaker, as the piezo speaker, needs a drive circuit. 
The circuit from figure \ref{fig:buzzer_circuit} was used again and the same frequency generator was used. 
In this setup the miniature speaker also produces clicks that was audible, but the sound produces was of significantly lower volume.
Which was also expected.  

\subsubsection*{Conclusion}
It was chosen to use the Piezo speaker with the developed drive circuit as it has the capability of producing a click sound and it produced a significantly greater volume than the miniature speaker in the same setup. 
Besides that it produces no electro magnetic noise as the miniature speaker does. 

\subsection{PWM Generation} % (fold)
\label{sub:pwm_generation}
The Raspberry Pi previously used has support only for one PWM channel.
In order to fully control the robot, four channels are required.
For this reason it was chosen to use an external PWM generator which can be communicated with through I2C.
Moving to an FPGA based platform, this is no longer necessary, as it is possible to add as many PWM channels as required in VHDL\@.
% subsection pwm_generation (end)

\subsection{Motor Driver} % (fold)
\label{sub:motor_driver}
The BD6222HFP was chosen as the motor driver in the previous project.
This is a full-bridge capable of continuously supplying sufficient power to drive the motors on the robot.
This driver chip is maintained in the new design.
% subsection motor_driver (end)

\subsection{Current Limitations} % (fold)
\label{sub:power_and_current_limitations}
A number of fuses were added to limit the current to the motors as well as the current draw from the battery.
The battery however, includes a ``safety board'' which limits the maximum possible current draw, making this fuse irrelevant.
The fuses on the motors can be excluded by instead running a check in software such that when a voltage is applied, the encoder signal must represent movement, or the voltage will be cut.
Allowing overcurrent for a short period will not damage the motors and therefore this approach is acceptable.
% subsection power_and_current_limitations (end)

\subsection{Electronics Board} % (fold)
\label{sub:electronics_board}
In order to accommodate the new electronics a new board will have to be designed.
The board must support a number of components and circuits, listed below.
\begin{itemize}
	\item \textbf{Motor controller:} The BD6222HFP, a full-bridge motor controller, is used to generate the drive signals for the motors.
	\item \textbf{Piezo:} Generating a click is done using a piezo transducer.
	A transducer with an internal drive circuit will be used.
	\item \textbf{5V DC/DC Converter:} A DC/DC converter will be used to generate the 5V rail necessary to drive the embedded platform.
	\item \textbf{Connections:} A number of connections has to be present on the board.
	Passthrough for the microphone add-in board.
	Passthrough for the motor encoders.
	Connection for battery.
	\item \textbf{Debug LEDs:} Support for four LEDs for debugging is required.
	The necessary drive circuitry must be added.
	\item \textbf{Microphone board:} A microphone board has been developed at SDU, supporting up to four SPH0641LU4H-1 digital microphones. The board must allow for interfacing with this microphone board.
\end{itemize}
\mikkel{Not true about the piezo.}
% subsection electronics_board (end)

\subsection{Power Calculations} % (fold)
\label{sub:power_calculations}
In the previous project it was chosen to use linear regulators to supply the 5 and 3.3V rails.
This project will analyse whether the added cost of using switch-mode converters is worthwile considering the possible extra battery life.
It will be analyzed how much power is drawn from the 5V- and battery voltage-rails and what the expected additional battery life will be when using a switch-mode converter.
It has been found that the DC/DC converter, \texttt{PTH08080WAH}, is suitable for this project and this component's specifications will be used for calculations. 

\subsubsection*{5V Rail (VCC)} % (fold)

% subsection subsection_name (end)
The following are powered by VCC:
\begin{itemize}
	\item Microphones including circuitry, $11$mA
	\item Zynqbased embedded platform, $500$mA
	\item Two motorcontrollers, $5$mA
	\item Six LEDs, $120$mA
\end{itemize}
The current estimation of the microphones, motorcontrollers and LEDs are based upon datasheet values. 
Whereas the current estimation for the Zynqbased platform is based upon current measurements from a previous project.
The current drawn of the Zynq is obviously dependent on the program running, the amount of logic in use, the number of GPIO in use and so on. All of these are unknown at time of writing.
500mA is by no means the maximum current draw that can be expected from a Zynqbased platform, as the MicroZed draws 1.7A at 85\% utilization \cite{microzed_hardware_guide}.

These estimates bring the total current draw of the VCC rail to $636$mA

\subsubsection*{Battery Voltage Rail (VBATT)} % (fold)

The motors will be running directly from VBATT.
To measure the current drawn by the motors and motorcontrollers a small test was conducted.
Three different voltages were applied to the motors using a laboratory power supply. 
Voltage and current measurements were made with the robot driving. 
The results are shown in table \ref{tab:motor_power}.

\begin{table}[h]
\centering
\caption{Voltage, current and power of motors and motorcontrollers measured at different velocities.}
\label{tab:motor_power}
\begin{tabular}{|l|c|c|c|}
\hline
\textbf{Velocity} & Voltage, [V]   & Current, [mA] & Power, [W]      \\ \hline
Low     	& 1  & 220 & 0.2 \\ \hline
Medium 		& 5  & 305 & 1.5 \\ \hline
High        & 8  & 350 & 2.8 \\ \hline
\end{tabular}
\end{table}

The piezo speaker is powered by VBATT as well, but it produces sound only for a few milliseconds at a time and in addition, with a very power.
For this reason, this current draw is neglected throughout the remainder of the discussion.

\subsubsection*{Power Dissipation} % (fold)
\label{sec:power_dissipation}
When using a linear voltage regulator the power dissipated in the component can be calculated using:
$$P_{LR} = (V_{in} - V_{out}) \cdot I_{load}$$

The power dissipation in the linear regulalator can be found using the current estimation at 5V and the fact that the battery has a nominal voltage of 7.4V:
$$P_{LR} = (7.4 [V] - 5.0 [V]) \cdot 0.636 [A] = 1.5 [W]$$

If a switching regulator is used the voltage conversion will have a significantly higher efficiency.
The \texttt{pth08080w} has a typical efficiency of $93.5\%$ at 5V \cite{pth08080}.
The power dissipated in the DC/DC converter would be:
$$P_{SW} = V_{in} \cdot I_{load} \cdot (1 - \eta) $$
$$P_{SW} = 7.4 [V] \cdot 0.636 [A] \cdot 0.065 = 0.3 [W]$$


The total power drawn at the 5V rail is:
$$P_{5V} = 0.636 [A] \cdot 5 [V] = 3.2 [W]$$

Based on the measurements of voltages and currents of the motors and motorcontrollers, the power dissipated were calculated to be as in table \ref{tab:motor_power}.
The total power usage can be calculated as the sum of the combined powers.
$$P_{T} = P_{reg} + P_{motor} + P_{5V}$$ 
Calculating the total power usage at low velocity and using a linear regulator yields:
$$P_{T,LR} = 1.5 + 0.2 + 3.2 = 4.9 [W]$$ 
The total power at the three velocities is calculated using a linear regulator or a DC/DC converter and is shown in table \ref{tab:drive_time}. 

\subsubsection*{Drive Time}
The amount of energy the battery can store should be calculated in order to calculate the drive time of the robot.
The batteries have a rated capacity of $2600$mAH \cite{battery}.
Converted to Joule: 
$$ E_{bat} = 2.6 [AH] \cdot 7.4 [V]  \cdot 3600 = 69.3 [kJ]$$
The drive time can then be calculated: 
$$ T = \frac{E_{bat}}{P_T} $$
Specifically for low velocity and using a linear regulator:
$$ T_{LR} = \frac{E_{bat}}{P_{T,LR}} = \frac{69.3 [kJ]}{4.9 [W]} = 14143 [s] $$
Which corresponds to 3.9h. The calculation is done for all scenarios and the results are shown in table \ref{tab:drive_time}.
Finally the extra drive time obtained by using a switching regulator rather than a linear regulator is calculated and shown in percent in table \ref{tab:drive_time}.

\begin{table}[h]
	\centering
	\caption{Total power and drive times using linear regulator or switching regulator. The obtained extra drive time by using a switching regulator is given in percent.}
	\label{tab:drive_time}
	\begin{tabular}{|l|c|c|c|c|c|}
		\hline
		\textbf{Velocity} & $P_{T,LR}$ [W]& $P_{T,SW}$ [W]& $T_{LR}$ [h]& $T_{SW}$ [h]& \% \textbf{more} $T$ \\ \hline
		Low       	& 4.9                         & 3.7                               & 3.9              & 5.2                      & 33                 \\ \hline
		Medium  	& 6.2                         & 5                                 & 3.1              & 3.8                      & 23                 \\ \hline
		High      	& 7.5                         & 6.3                               & 2.6              & 3.1                      & 19                 \\ \hline
	\end{tabular}
\end{table}

Using a switching regulator yields a minimum of 19\% more drive time compared to using a linear regulator.
On the basis of the results it is concluded that using a switching regulator is worth the additional cost.

Note that these results are very conservative estimates.
The more power is being drawn from VCC, the more the benefits if the switch-mode converter would show. 
% subsection power_calculations (end)
% section analysis (end)
