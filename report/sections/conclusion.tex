%!TEX root = ../main.tex
\section{Conclusion}
This project develops on the work done in the bachelor thesis ``Investigating Bio-inspired Object Avoidance in a Swarm of Mobile Robots'' with the main goal of replacing the Raspberry Pi with a Zynq based platform.
It was decided to keep the mechanical platform, battery, motors, motor drivers.
Fuses were excluded as the batteries include a "safety boards" and software can be used to protect the motors.
Tests were made to find the sound producer best suited for making a clicking noise on the platform.
It was found that a piezo speaker is the best choice, but it was not included on the board to allow for further testing, before making the final decision.
A switch-mode DC/DC converter is included as tests and calculations showed that a minimum of 19\% extra drive time is gained when using this component instead of a linear voltage regulator.
It was decided to use the MicroZed Zynq platform rather the Zynqberry or the Zybo.
The MicroZed was chosen as it has plenty of I/O pins, has a solid documentation and it is within the confines of the robot platform.
When using the MicroZed a carrier card is needed to power the I/O banks and give access to I/O pins.
\\~\\
It was decided to design such a carrier card and include all electronic for the project on it. 
The PTH08080WAH converter was chosen and test showed that at least 70 minutes of use can be expected from the robot.
It was chosen to supply all I/O banks with 3.3V and circuits were designed to ensure correct power up and power down sequences. 
Four pairs of anti-aliasing filters were added to enable use of the on-chip ADCs. 
The carrier card layout was done using Altium Designer and it was decided to have the board manufactured by a professional outlet to allow for a multilayer board, small feature size and plated vias.
\\~\\
After receiving and soldering the designed carrier cards, it was found that the board was not functioning correctly.
Thorough physical debugging and fixing of errors led to a correctly functioning board. 
Correct power up and power down sequences were verified by measuring voltages on the included power header. 
Verification of main features of the board, such as I/O pins, was made by performing small "Hello World" style programs in \texttt{C} and \texttt{VHDL}.
\\~\\
In hindsight, it is clear to the authors that the decision to develop a MicroZed carrier card resulted in many hours being spent on this, rather than developing a robot, but it led to a very valuable exercise in the development of a professional layout of PCB and, hopefully¸ the beginning of a platform that can be used by future students.
\\~\\
In addition to the work enclosed herein, documentation such as pinout, schematic and layout, including the entire Altium project, can all be found on the \texttt{SDU-Embedded:swarm\_bot} github \cite{github}.